% Options for packages loaded elsewhere
\PassOptionsToPackage{unicode}{hyperref}
\PassOptionsToPackage{hyphens}{url}
%
\documentclass[
]{article}
\usepackage{lmodern}
\usepackage{amssymb,amsmath}
\usepackage{ifxetex,ifluatex}
\ifnum 0\ifxetex 1\fi\ifluatex 1\fi=0 % if pdftex
  \usepackage[T1]{fontenc}
  \usepackage[utf8]{inputenc}
  \usepackage{textcomp} % provide euro and other symbols
\else % if luatex or xetex
  \usepackage{unicode-math}
  \defaultfontfeatures{Scale=MatchLowercase}
  \defaultfontfeatures[\rmfamily]{Ligatures=TeX,Scale=1}
\fi
% Use upquote if available, for straight quotes in verbatim environments
\IfFileExists{upquote.sty}{\usepackage{upquote}}{}
\IfFileExists{microtype.sty}{% use microtype if available
  \usepackage[]{microtype}
  \UseMicrotypeSet[protrusion]{basicmath} % disable protrusion for tt fonts
}{}
\makeatletter
\@ifundefined{KOMAClassName}{% if non-KOMA class
  \IfFileExists{parskip.sty}{%
    \usepackage{parskip}
  }{% else
    \setlength{\parindent}{0pt}
    \setlength{\parskip}{6pt plus 2pt minus 1pt}}
}{% if KOMA class
  \KOMAoptions{parskip=half}}
\makeatother
\usepackage{xcolor}
\IfFileExists{xurl.sty}{\usepackage{xurl}}{} % add URL line breaks if available
\IfFileExists{bookmark.sty}{\usepackage{bookmark}}{\usepackage{hyperref}}
\hypersetup{
  pdftitle={Algunos Ejemplos de Inferencias sobre Medias},
  hidelinks,
  pdfcreator={LaTeX via pandoc}}
\urlstyle{same} % disable monospaced font for URLs
\usepackage[margin=1in]{geometry}
\usepackage{color}
\usepackage{fancyvrb}
\newcommand{\VerbBar}{|}
\newcommand{\VERB}{\Verb[commandchars=\\\{\}]}
\DefineVerbatimEnvironment{Highlighting}{Verbatim}{commandchars=\\\{\}}
% Add ',fontsize=\small' for more characters per line
\usepackage{framed}
\definecolor{shadecolor}{RGB}{248,248,248}
\newenvironment{Shaded}{\begin{snugshade}}{\end{snugshade}}
\newcommand{\AlertTok}[1]{\textcolor[rgb]{0.94,0.16,0.16}{#1}}
\newcommand{\AnnotationTok}[1]{\textcolor[rgb]{0.56,0.35,0.01}{\textbf{\textit{#1}}}}
\newcommand{\AttributeTok}[1]{\textcolor[rgb]{0.77,0.63,0.00}{#1}}
\newcommand{\BaseNTok}[1]{\textcolor[rgb]{0.00,0.00,0.81}{#1}}
\newcommand{\BuiltInTok}[1]{#1}
\newcommand{\CharTok}[1]{\textcolor[rgb]{0.31,0.60,0.02}{#1}}
\newcommand{\CommentTok}[1]{\textcolor[rgb]{0.56,0.35,0.01}{\textit{#1}}}
\newcommand{\CommentVarTok}[1]{\textcolor[rgb]{0.56,0.35,0.01}{\textbf{\textit{#1}}}}
\newcommand{\ConstantTok}[1]{\textcolor[rgb]{0.00,0.00,0.00}{#1}}
\newcommand{\ControlFlowTok}[1]{\textcolor[rgb]{0.13,0.29,0.53}{\textbf{#1}}}
\newcommand{\DataTypeTok}[1]{\textcolor[rgb]{0.13,0.29,0.53}{#1}}
\newcommand{\DecValTok}[1]{\textcolor[rgb]{0.00,0.00,0.81}{#1}}
\newcommand{\DocumentationTok}[1]{\textcolor[rgb]{0.56,0.35,0.01}{\textbf{\textit{#1}}}}
\newcommand{\ErrorTok}[1]{\textcolor[rgb]{0.64,0.00,0.00}{\textbf{#1}}}
\newcommand{\ExtensionTok}[1]{#1}
\newcommand{\FloatTok}[1]{\textcolor[rgb]{0.00,0.00,0.81}{#1}}
\newcommand{\FunctionTok}[1]{\textcolor[rgb]{0.00,0.00,0.00}{#1}}
\newcommand{\ImportTok}[1]{#1}
\newcommand{\InformationTok}[1]{\textcolor[rgb]{0.56,0.35,0.01}{\textbf{\textit{#1}}}}
\newcommand{\KeywordTok}[1]{\textcolor[rgb]{0.13,0.29,0.53}{\textbf{#1}}}
\newcommand{\NormalTok}[1]{#1}
\newcommand{\OperatorTok}[1]{\textcolor[rgb]{0.81,0.36,0.00}{\textbf{#1}}}
\newcommand{\OtherTok}[1]{\textcolor[rgb]{0.56,0.35,0.01}{#1}}
\newcommand{\PreprocessorTok}[1]{\textcolor[rgb]{0.56,0.35,0.01}{\textit{#1}}}
\newcommand{\RegionMarkerTok}[1]{#1}
\newcommand{\SpecialCharTok}[1]{\textcolor[rgb]{0.00,0.00,0.00}{#1}}
\newcommand{\SpecialStringTok}[1]{\textcolor[rgb]{0.31,0.60,0.02}{#1}}
\newcommand{\StringTok}[1]{\textcolor[rgb]{0.31,0.60,0.02}{#1}}
\newcommand{\VariableTok}[1]{\textcolor[rgb]{0.00,0.00,0.00}{#1}}
\newcommand{\VerbatimStringTok}[1]{\textcolor[rgb]{0.31,0.60,0.02}{#1}}
\newcommand{\WarningTok}[1]{\textcolor[rgb]{0.56,0.35,0.01}{\textbf{\textit{#1}}}}
\usepackage{graphicx,grffile}
\makeatletter
\def\maxwidth{\ifdim\Gin@nat@width>\linewidth\linewidth\else\Gin@nat@width\fi}
\def\maxheight{\ifdim\Gin@nat@height>\textheight\textheight\else\Gin@nat@height\fi}
\makeatother
% Scale images if necessary, so that they will not overflow the page
% margins by default, and it is still possible to overwrite the defaults
% using explicit options in \includegraphics[width, height, ...]{}
\setkeys{Gin}{width=\maxwidth,height=\maxheight,keepaspectratio}
% Set default figure placement to htbp
\makeatletter
\def\fps@figure{htbp}
\makeatother
\setlength{\emergencystretch}{3em} % prevent overfull lines
\providecommand{\tightlist}{%
  \setlength{\itemsep}{0pt}\setlength{\parskip}{0pt}}
\setcounter{secnumdepth}{-\maxdimen} % remove section numbering

\title{Algunos Ejemplos de Inferencias sobre Medias}
\author{}
\date{\vspace{-2.5em}}

\begin{document}
\maketitle

\hypertarget{ejemplo-1}{%
\subsection{Ejemplo 1}\label{ejemplo-1}}

\begin{Shaded}
\begin{Highlighting}[]
\KeywordTok{rm}\NormalTok{(}\DataTypeTok{list=}\KeywordTok{ls}\NormalTok{())}

\NormalTok{X <-}\StringTok{ }\KeywordTok{matrix}\NormalTok{(}\KeywordTok{c}\NormalTok{(}\DecValTok{6}\NormalTok{,}\DecValTok{10}\NormalTok{,}\DecValTok{8}\NormalTok{,}
              \DecValTok{9}\NormalTok{,}\DecValTok{6}\NormalTok{,}\DecValTok{3}\NormalTok{),}\DataTypeTok{byrow=}\OtherTok{FALSE}\NormalTok{,}\DataTypeTok{ncol=}\DecValTok{2}\NormalTok{)}

\NormalTok{mu0 <-}\StringTok{ }\KeywordTok{matrix}\NormalTok{(}\KeywordTok{c}\NormalTok{(}\DecValTok{9}\NormalTok{,}\DecValTok{5}\NormalTok{),}\DataTypeTok{ncol=}\DecValTok{1}\NormalTok{)}

\NormalTok{n <-}\StringTok{ }\KeywordTok{nrow}\NormalTok{(X)}
\NormalTok{p <-}\StringTok{ }\KeywordTok{ncol}\NormalTok{(X)}

\NormalTok{(X.mean <-}\StringTok{ }\KeywordTok{t}\NormalTok{(}\KeywordTok{matrix}\NormalTok{(}\DecValTok{1}\NormalTok{,}\DataTypeTok{ncol=}\NormalTok{n) }\OperatorTok\StringTok{ }\NormalTok{X)}\OperatorTok{/}\NormalTok{n)}
\end{Highlighting}
\end{Shaded}

\begin{verbatim}
##      [,1]
## [1,]    8
## [2,]    6
\end{verbatim}

\begin{Shaded}
\begin{Highlighting}[]
\NormalTok{(D <-}\StringTok{ }\NormalTok{X }\OperatorTok{-}\StringTok{ }\KeywordTok{matrix}\NormalTok{(}\DecValTok{1}\NormalTok{,}\DataTypeTok{nrow=}\NormalTok{n) }\OperatorTok\StringTok{ }\KeywordTok{t}\NormalTok{(X.mean))}
\end{Highlighting}
\end{Shaded}

\begin{verbatim}
##      [,1] [,2]
## [1,]   -2    3
## [2,]    2    0
## [3,]    0   -3
\end{verbatim}

\begin{Shaded}
\begin{Highlighting}[]
\NormalTok{(S <-}\StringTok{ }\NormalTok{(n}\DecValTok{-1}\NormalTok{)}\OperatorTok{^}\NormalTok{(}\OperatorTok{-}\DecValTok{1}\NormalTok{) }\OperatorTok{*}\StringTok{ }\KeywordTok{t}\NormalTok{(D)}\OperatorTok\NormalTok{D)}
\end{Highlighting}
\end{Shaded}

\begin{verbatim}
##      [,1] [,2]
## [1,]    4   -3
## [2,]   -3    9
\end{verbatim}

\begin{Shaded}
\begin{Highlighting}[]
\NormalTok{Sinv <-}\StringTok{ }\KeywordTok{solve}\NormalTok{(S)}

\NormalTok{T2 <-}\StringTok{ }\NormalTok{n }\OperatorTok{*}\StringTok{ }\KeywordTok{t}\NormalTok{(X.mean}\OperatorTok{-}\NormalTok{mu0) }\OperatorTok\StringTok{ }\NormalTok{Sinv }\OperatorTok\StringTok{ }\NormalTok{(X.mean}\OperatorTok{-}\NormalTok{mu0)}

\CommentTok{# T^2 esta distribuida como ((n-1)*p)/(n-p)*qf(1-alpha,p,n-p)}

\NormalTok{(}\DecValTok{3-1}\NormalTok{)}\OperatorTok{*}\DecValTok{2}\OperatorTok{/}\NormalTok{(}\DecValTok{3-2}\NormalTok{)}\OperatorTok{*}\KeywordTok{qf}\NormalTok{(}\DecValTok{1}\FloatTok{-0.05}\NormalTok{,}\DecValTok{2}\NormalTok{,}\DecValTok{3-2}\NormalTok{)}
\end{Highlighting}
\end{Shaded}

\begin{verbatim}
## [1] 798
\end{verbatim}

\hypertarget{ejemplo-sweat-data}{%
\subsection{Ejemplo Sweat Data}\label{ejemplo-sweat-data}}

\hypertarget{probando-un-vector-medio-multivariado-con-t2}{%
\subsubsection{\texorpdfstring{Probando un vector medio multivariado con
\(T^{2}\)}{Probando un vector medio multivariado con T\^{}\{2\}}}\label{probando-un-vector-medio-multivariado-con-t2}}

Se analizó la transpiración de 20 mujeres sanas. Se midieron tres
componentes, \(X_{1} =\) tasa de sudoración, \(X_{2}=\) contenido de
sodio y \(X_{3} =\) contenido de potasio, y se presentan los resultados
(sweat dataset).

\begin{Shaded}
\begin{Highlighting}[]
\KeywordTok{rm}\NormalTok{(}\DataTypeTok{list=}\KeywordTok{ls}\NormalTok{())}

\NormalTok{dat <-}\StringTok{ }\KeywordTok{read.table}\NormalTok{(}\StringTok{"sweat.DAT"}\NormalTok{)}
\KeywordTok{names}\NormalTok{(dat)[}\DecValTok{1}\NormalTok{] <-}\StringTok{ "Sweat rate"}
\KeywordTok{names}\NormalTok{(dat)[}\DecValTok{2}\NormalTok{] <-}\StringTok{ "Sodium"}
\KeywordTok{names}\NormalTok{(dat)[}\DecValTok{3}\NormalTok{] <-}\StringTok{ "Potassium"}
\KeywordTok{head}\NormalTok{(dat)}
\end{Highlighting}
\end{Shaded}

\begin{verbatim}
##   Sweat rate Sodium Potassium
## 1        3.7   48.5       9.3
## 2        5.7   65.1       8.0
## 3        3.8   47.2      10.9
## 4        3.2   53.2      12.0
## 5        3.1   55.5       9.7
## 6        4.6   36.1       7.9
\end{verbatim}

Pruebe la hipótesis \(H_{0}: \boldsymbol {\mu} ^ {\prime} = [4,50,10]\)
contra \(H_{1}: \boldsymbol {\mu} ^ {\prime} \neq [4,50,10]\) al nivel
de significancia \(\alpha = .10\).

Para una prueba de la hipótesis
\(H_{0}: \boldsymbol {\mu} = \boldsymbol {\mu} _{0}\) versus
\(H_{1}: \boldsymbol { \mu} \neq \boldsymbol {\mu} _{0}.\) En el nivel
de significancia \(\alpha\), rechazamos \(H_{0}\) a favor de \(H_{1}\)
si el \[
T ^ {2} = n \left (\overline {\mathbf {x}} - \boldsymbol {\mu} _{0} \right) ^ {\prime} \mathbf {S} ^ {- 1} \left (\overline {\mathbf {x}} - \boldsymbol {\mu} _{0} \right)> \frac {(n-1) p} {(np)} F_{p, np} (\alpha)
\] Primero determinamos el estadístico \(T^2\):

\begin{Shaded}
\begin{Highlighting}[]
\NormalTok{X <-}\StringTok{ }\KeywordTok{as.matrix}\NormalTok{(dat)}
\NormalTok{n <-}\StringTok{ }\KeywordTok{nrow}\NormalTok{(X)}
\NormalTok{p <-}\StringTok{ }\KeywordTok{ncol}\NormalTok{(X)}
\NormalTok{X.mean <-}\StringTok{ }\KeywordTok{t}\NormalTok{(}\KeywordTok{matrix}\NormalTok{(}\DecValTok{1}\NormalTok{,}\DataTypeTok{ncol=}\NormalTok{n) }\OperatorTok\StringTok{ }\NormalTok{X)}\OperatorTok{/}\NormalTok{n}
\NormalTok{D <-}\StringTok{ }\NormalTok{X }\OperatorTok{-}\StringTok{ }\KeywordTok{matrix}\NormalTok{(}\DecValTok{1}\NormalTok{,}\DataTypeTok{nrow=}\NormalTok{n) }\OperatorTok\StringTok{ }\KeywordTok{t}\NormalTok{(X.mean)}
\NormalTok{S <-}\StringTok{ }\NormalTok{(n}\DecValTok{-1}\NormalTok{)}\OperatorTok{^}\NormalTok{(}\OperatorTok{-}\DecValTok{1}\NormalTok{) }\OperatorTok{*}\StringTok{ }\KeywordTok{t}\NormalTok{(D)}\OperatorTok\NormalTok{D}
\NormalTok{Sinv <-}\StringTok{ }\KeywordTok{solve}\NormalTok{(S)}

\NormalTok{mu =}\StringTok{ }\KeywordTok{c}\NormalTok{(}\DecValTok{4}\NormalTok{,}\DecValTok{50}\NormalTok{,}\DecValTok{10}\NormalTok{)}

\NormalTok{(}\DataTypeTok{T2 =}\NormalTok{ n}\OperatorTok{*}\KeywordTok{t}\NormalTok{(X.mean}\OperatorTok{-}\NormalTok{mu)}\OperatorTok\NormalTok{Sinv}\OperatorTok\NormalTok{(X.mean}\OperatorTok{-}\NormalTok{mu))}
\end{Highlighting}
\end{Shaded}

\begin{verbatim}
##          [,1]
## [1,] 9.738773
\end{verbatim}

Y luego calculamos el valor crítico:

\begin{Shaded}
\begin{Highlighting}[]
\NormalTok{alpha <-}\StringTok{ }\FloatTok{0.1} \CommentTok{# Nivel de significancia}

\NormalTok{(critical.value <-}\StringTok{ }\NormalTok{(p}\OperatorTok{*}\NormalTok{(n}\DecValTok{-1}\NormalTok{))}\OperatorTok{/}\NormalTok{(n}\OperatorTok{-}\NormalTok{p)}\OperatorTok{*}\KeywordTok{qf}\NormalTok{(}\DecValTok{1}\OperatorTok{-}\NormalTok{alpha,p,n}\OperatorTok{-}\NormalTok{p))}
\end{Highlighting}
\end{Shaded}

\begin{verbatim}
## [1] 8.172573
\end{verbatim}

Y comprobamos si el estadístico es mayor o menor que el valor crítico:

\begin{Shaded}
\begin{Highlighting}[]
\NormalTok{(T2 }\OperatorTok{>}\StringTok{ }\NormalTok{critical.value)}
\end{Highlighting}
\end{Shaded}

\begin{verbatim}
##      [,1]
## [1,] TRUE
\end{verbatim}

Y por lo tanto al nivel de un \(\alpha=0.1\) rechazamos la hipótesis
nula.

\hypertarget{ejemplo-de-radiaciuxf3n}{%
\subsection{Ejemplo de Radiación}\label{ejemplo-de-radiaciuxf3n}}

El gobierno federal requiere que el departamento de control de calidad
de un fabricante de hornos microondas controle la cantidad de radiación
emitida cuando las puertas de los hornos están cerradas. Se realizaron
observaciones de la radiación emitida a través de puertas cerradas de
\(n = 42\) en hornos seleccionados al azar.Las mediciones de radiación
también se registraron a través de las puertas abiertas de los hornos
microondas.

Leemos los datos

\begin{Shaded}
\begin{Highlighting}[]
\KeywordTok{rm}\NormalTok{(}\DataTypeTok{list=}\KeywordTok{ls}\NormalTok{())}

\NormalTok{alpha <-}\StringTok{ }\FloatTok{0.05}

\NormalTok{dat.closed <-}\StringTok{ }\KeywordTok{read.table}\NormalTok{(}\StringTok{"closed.DAT"}\NormalTok{) }\CommentTok{# Door closed}
\NormalTok{dat.open <-}\StringTok{ }\KeywordTok{read.table}\NormalTok{(}\StringTok{"open.DAT"}\NormalTok{) }\CommentTok{# Door Open}
\NormalTok{dat <-}\StringTok{ }\KeywordTok{data.frame}\NormalTok{(}\DataTypeTok{closed=}\NormalTok{dat.closed[,}\DecValTok{1}\NormalTok{]}\OperatorTok{^}\NormalTok{.}\DecValTok{25}\NormalTok{, }
                  \DataTypeTok{open=}\NormalTok{dat.open[,}\DecValTok{1}\NormalTok{]}\OperatorTok{^}\NormalTok{.}\DecValTok{25}\NormalTok{)}

\KeywordTok{head}\NormalTok{(dat)}
\end{Highlighting}
\end{Shaded}

\begin{verbatim}
##      closed      open
## 1 0.6223330 0.7400828
## 2 0.5477226 0.5477226
## 3 0.6513556 0.7400828
## 4 0.5623413 0.5623413
## 5 0.4728708 0.5623413
## 6 0.5885662 0.5885662
\end{verbatim}

Realizamos la inferencia para una prueba de hipótesis de que la media es
\(\mu=(0.562,0.589)\)

\begin{Shaded}
\begin{Highlighting}[]
\NormalTok{X <-}\StringTok{ }\KeywordTok{as.matrix}\NormalTok{(dat)}
\NormalTok{n <-}\StringTok{ }\KeywordTok{nrow}\NormalTok{(X)}
\NormalTok{p <-}\StringTok{ }\KeywordTok{ncol}\NormalTok{(X)}
\NormalTok{X.mean <-}\StringTok{ }\KeywordTok{t}\NormalTok{(}\KeywordTok{matrix}\NormalTok{(}\DecValTok{1}\NormalTok{,}\DataTypeTok{ncol=}\NormalTok{n) }\OperatorTok\StringTok{ }\NormalTok{X)}\OperatorTok{/}\NormalTok{n}
\NormalTok{D <-}\StringTok{ }\NormalTok{X }\OperatorTok{-}\StringTok{ }\KeywordTok{matrix}\NormalTok{(}\DecValTok{1}\NormalTok{,}\DataTypeTok{nrow=}\NormalTok{n) }\OperatorTok\StringTok{ }\KeywordTok{t}\NormalTok{(X.mean)}
\NormalTok{S <-}\StringTok{ }\NormalTok{(n}\DecValTok{-1}\NormalTok{)}\OperatorTok{^}\NormalTok{(}\OperatorTok{-}\DecValTok{1}\NormalTok{) }\OperatorTok{*}\StringTok{ }\KeywordTok{t}\NormalTok{(D)}\OperatorTok\NormalTok{D}
\NormalTok{Sinv <-}\StringTok{ }\KeywordTok{solve}\NormalTok{(S)}

\NormalTok{inRegion <-}\StringTok{ }\ControlFlowTok{function}\NormalTok{(mu, X.mean, Sinv, n, alpha)\{}
\NormalTok{  critical.value <-}\StringTok{ }\NormalTok{(p}\OperatorTok{*}\NormalTok{(n}\DecValTok{-1}\NormalTok{))}\OperatorTok{/}\NormalTok{(n}\OperatorTok{-}\NormalTok{p)}\OperatorTok{*}\KeywordTok{qf}\NormalTok{(}\DecValTok{1}\OperatorTok{-}\NormalTok{alpha,p,n}\OperatorTok{-}\NormalTok{p)}
  \KeywordTok{return}\NormalTok{(n}\OperatorTok{*}\KeywordTok{t}\NormalTok{(X.mean}\OperatorTok{-}\NormalTok{mu)}\OperatorTok\NormalTok{Sinv}\OperatorTok\NormalTok{(X.mean}\OperatorTok{-}\NormalTok{mu) }\OperatorTok{>}\StringTok{ }\NormalTok{critical.value)}
\NormalTok{\}}

\KeywordTok{inRegion}\NormalTok{(}\DataTypeTok{mu=}\KeywordTok{matrix}\NormalTok{(}\KeywordTok{c}\NormalTok{(}\FloatTok{0.562}\NormalTok{,}\FloatTok{0.589}\NormalTok{),}
                   \DataTypeTok{ncol=}\DecValTok{1}\NormalTok{),}
         \DataTypeTok{X.mean=}\NormalTok{X.mean,}
         \DataTypeTok{Sinv=}\NormalTok{Sinv,}
         \DataTypeTok{n=}\NormalTok{n,}
         \DataTypeTok{alpha=}\NormalTok{alpha)}
\end{Highlighting}
\end{Shaded}

\begin{verbatim}
##       [,1]
## [1,] FALSE
\end{verbatim}

Y graficamos la elipse respectiva:

\begin{Shaded}
\begin{Highlighting}[]
\KeywordTok{library}\NormalTok{(plotrix)}
\NormalTok{angle <-}\StringTok{ }\KeywordTok{atan}\NormalTok{(}\KeywordTok{eigen}\NormalTok{(S)}\OperatorTok{$}\NormalTok{vectors[}\DecValTok{2}\NormalTok{,}\DecValTok{1}\NormalTok{]}\OperatorTok{/}\KeywordTok{eigen}\NormalTok{(S)}\OperatorTok{$}\NormalTok{vectors[}\DecValTok{1}\NormalTok{,}\DecValTok{1}\NormalTok{]) }
\KeywordTok{plot}\NormalTok{(}\DecValTok{0}\NormalTok{,}\DataTypeTok{pch=}\StringTok{''}\NormalTok{,}\DataTypeTok{ylab=}\StringTok{''}\NormalTok{,}\DataTypeTok{xlab=}\StringTok{''}\NormalTok{,}\DataTypeTok{xlim=}\KeywordTok{c}\NormalTok{(}\FloatTok{0.5}\NormalTok{,}\FloatTok{0.65}\NormalTok{),}\DataTypeTok{ylim=}\KeywordTok{c}\NormalTok{(}\FloatTok{0.55}\NormalTok{,}\FloatTok{0.65}\NormalTok{))}
\NormalTok{axis1 <-}\StringTok{ }\KeywordTok{sqrt}\NormalTok{(}\KeywordTok{eigen}\NormalTok{(S)}\OperatorTok{$}\NormalTok{values[}\DecValTok{1}\NormalTok{])}\OperatorTok{*}
\StringTok{  }\KeywordTok{sqrt}\NormalTok{(}
\NormalTok{    (p}\OperatorTok{*}\NormalTok{(n}\DecValTok{-1}\NormalTok{))}\OperatorTok{/}
\StringTok{      }\NormalTok{(n}\OperatorTok{*}\NormalTok{(n}\OperatorTok{-}\NormalTok{p))}\OperatorTok{*}\StringTok{ }\CommentTok{# no es el mismo valor critico que antes }
\StringTok{      }\KeywordTok{qf}\NormalTok{(}\DecValTok{1}\OperatorTok{-}\NormalTok{alpha,p,n}\OperatorTok{-}\NormalTok{p)) }
\NormalTok{axis2 <-}\StringTok{ }\KeywordTok{sqrt}\NormalTok{(}\KeywordTok{eigen}\NormalTok{(S)}\OperatorTok{$}\NormalTok{values[}\DecValTok{2}\NormalTok{])}\OperatorTok{*}
\StringTok{  }\KeywordTok{sqrt}\NormalTok{((p}\OperatorTok{*}\NormalTok{(n}\DecValTok{-1}\NormalTok{))}\OperatorTok{/}\NormalTok{(n}\OperatorTok{*}\NormalTok{(n}\OperatorTok{-}\NormalTok{p))}\OperatorTok{*}\KeywordTok{qf}\NormalTok{(}\DecValTok{1}\OperatorTok{-}\NormalTok{alpha,p,n}\OperatorTok{-}\NormalTok{p))}
\NormalTok{lengths <-}\StringTok{ }\KeywordTok{c}\NormalTok{(axis1,axis2)}
\KeywordTok{draw.ellipse}\NormalTok{(}\DataTypeTok{x=}\NormalTok{X.mean[}\DecValTok{1}\NormalTok{,}\DecValTok{1}\NormalTok{],}\DataTypeTok{y=}\NormalTok{X.mean[}\DecValTok{2}\NormalTok{,}\DecValTok{1}\NormalTok{],}\DataTypeTok{a=}\NormalTok{lengths[}\DecValTok{1}\NormalTok{],}\DataTypeTok{b=}\NormalTok{lengths[}\DecValTok{2}\NormalTok{],}\DataTypeTok{angle=}\NormalTok{angle,}\DataTypeTok{deg=}\OtherTok{FALSE}\NormalTok{)}
\end{Highlighting}
\end{Shaded}

\includegraphics{EjemplosR_Inferencias_medias_files/figure-latex/unnamed-chunk-8-1.pdf}

\begin{Shaded}
\begin{Highlighting}[]
\CommentTok{# sqrt(eigen(S)$values[1])/sqrt(eigen(S)$values[2])}
\end{Highlighting}
\end{Shaded}

Ahora calculemos los intervalos de confianza simultaneos de \(T^2\) para
las dos componentes de la media:

\begin{Shaded}
\begin{Highlighting}[]
\NormalTok{a1 <-}\StringTok{ }\KeywordTok{matrix}\NormalTok{(}\KeywordTok{c}\NormalTok{(}\DecValTok{1}\NormalTok{,}\DecValTok{0}\NormalTok{),}\DataTypeTok{ncol=}\DecValTok{1}\NormalTok{)}
\NormalTok{a2 <-}\StringTok{ }\KeywordTok{matrix}\NormalTok{(}\KeywordTok{c}\NormalTok{(}\DecValTok{0}\NormalTok{,}\DecValTok{1}\NormalTok{),}\DataTypeTok{ncol=}\DecValTok{1}\NormalTok{)}

\NormalTok{l1 <-}\StringTok{ }\NormalTok{X.mean[}\DecValTok{1}\NormalTok{,}\DecValTok{1}\NormalTok{] }\OperatorTok{-}\StringTok{ }\KeywordTok{sqrt}\NormalTok{((p}\OperatorTok{*}\NormalTok{(n}\DecValTok{-1}\NormalTok{))}\OperatorTok{/}\NormalTok{(n}\OperatorTok{*}\NormalTok{(n}\OperatorTok{-}\NormalTok{p))}\OperatorTok{*}\KeywordTok{qf}\NormalTok{(}\DecValTok{1}\OperatorTok{-}\NormalTok{alpha,p,n}\OperatorTok{-}\NormalTok{p)}\OperatorTok{*}\KeywordTok{t}\NormalTok{(a1)}\OperatorTok\NormalTok{S}\OperatorTok\NormalTok{a1)}
\NormalTok{u1 <-}\StringTok{ }\NormalTok{X.mean[}\DecValTok{1}\NormalTok{,}\DecValTok{1}\NormalTok{] }\OperatorTok{+}\StringTok{ }\KeywordTok{sqrt}\NormalTok{((p}\OperatorTok{*}\NormalTok{(n}\DecValTok{-1}\NormalTok{))}\OperatorTok{/}\NormalTok{(n}\OperatorTok{*}\NormalTok{(n}\OperatorTok{-}\NormalTok{p))}\OperatorTok{*}\KeywordTok{qf}\NormalTok{(}\DecValTok{1}\OperatorTok{-}\NormalTok{alpha,p,n}\OperatorTok{-}\NormalTok{p)}\OperatorTok{*}\KeywordTok{t}\NormalTok{(a1)}\OperatorTok\NormalTok{S}\OperatorTok\NormalTok{a1)}

\NormalTok{l2 <-}\StringTok{ }\NormalTok{X.mean[}\DecValTok{2}\NormalTok{,}\DecValTok{1}\NormalTok{] }\OperatorTok{-}\StringTok{ }\KeywordTok{sqrt}\NormalTok{((p}\OperatorTok{*}\NormalTok{(n}\DecValTok{-1}\NormalTok{))}\OperatorTok{/}\NormalTok{(n}\OperatorTok{*}\NormalTok{(n}\OperatorTok{-}\NormalTok{p))}\OperatorTok{*}\KeywordTok{qf}\NormalTok{(}\DecValTok{1}\OperatorTok{-}\NormalTok{alpha,p,n}\OperatorTok{-}\NormalTok{p)}\OperatorTok{*}\KeywordTok{t}\NormalTok{(a2)}\OperatorTok\NormalTok{S}\OperatorTok\NormalTok{a2)}
\NormalTok{u2 <-}\StringTok{ }\NormalTok{X.mean[}\DecValTok{2}\NormalTok{,}\DecValTok{1}\NormalTok{] }\OperatorTok{+}\StringTok{ }\KeywordTok{sqrt}\NormalTok{((p}\OperatorTok{*}\NormalTok{(n}\DecValTok{-1}\NormalTok{))}\OperatorTok{/}\NormalTok{(n}\OperatorTok{*}\NormalTok{(n}\OperatorTok{-}\NormalTok{p))}\OperatorTok{*}\KeywordTok{qf}\NormalTok{(}\DecValTok{1}\OperatorTok{-}\NormalTok{alpha,p,n}\OperatorTok{-}\NormalTok{p)}\OperatorTok{*}\KeywordTok{t}\NormalTok{(a2)}\OperatorTok\NormalTok{S}\OperatorTok\NormalTok{a2)}

\NormalTok{l1;u1}
\end{Highlighting}
\end{Shaded}

\begin{verbatim}
##           [,1]
## [1,] 0.5166803
\end{verbatim}

\begin{verbatim}
##           [,1]
## [1,] 0.6118347
\end{verbatim}

\begin{Shaded}
\begin{Highlighting}[]
\NormalTok{l2;u2}
\end{Highlighting}
\end{Shaded}

\begin{verbatim}
##           [,1]
## [1,] 0.5550817
\end{verbatim}

\begin{verbatim}
##           [,1]
## [1,] 0.6508807
\end{verbatim}

\begin{Shaded}
\begin{Highlighting}[]
\KeywordTok{plot}\NormalTok{(}\DecValTok{0}\NormalTok{,}\DataTypeTok{pch=}\StringTok{''}\NormalTok{,}\DataTypeTok{ylab=}\StringTok{''}\NormalTok{,}\DataTypeTok{xlab=}\StringTok{''}\NormalTok{,}\DataTypeTok{xlim=}\KeywordTok{c}\NormalTok{(}\FloatTok{0.5}\NormalTok{,}\FloatTok{0.65}\NormalTok{),}\DataTypeTok{ylim=}\KeywordTok{c}\NormalTok{(}\FloatTok{0.55}\NormalTok{,}\FloatTok{0.65}\NormalTok{))}
\KeywordTok{draw.ellipse}\NormalTok{(}\DataTypeTok{x=}\NormalTok{X.mean[}\DecValTok{1}\NormalTok{,}\DecValTok{1}\NormalTok{],}\DataTypeTok{y=}\NormalTok{X.mean[}\DecValTok{2}\NormalTok{,}\DecValTok{1}\NormalTok{],}\DataTypeTok{a=}\NormalTok{lengths[}\DecValTok{1}\NormalTok{],}\DataTypeTok{b=}\NormalTok{lengths[}\DecValTok{2}\NormalTok{],}\DataTypeTok{angle=}\NormalTok{angle,}\DataTypeTok{deg=}\OtherTok{FALSE}\NormalTok{)}

\KeywordTok{abline}\NormalTok{(}\DataTypeTok{v=}\NormalTok{l1,}\DataTypeTok{lty=}\DecValTok{3}\NormalTok{);}\KeywordTok{abline}\NormalTok{(}\DataTypeTok{v=}\NormalTok{u1,}\DataTypeTok{lty=}\DecValTok{3}\NormalTok{)}
\KeywordTok{abline}\NormalTok{(}\DataTypeTok{h=}\NormalTok{l2,}\DataTypeTok{lty=}\DecValTok{3}\NormalTok{);}\KeywordTok{abline}\NormalTok{(}\DataTypeTok{h=}\NormalTok{u2,}\DataTypeTok{lty=}\DecValTok{3}\NormalTok{)}
\end{Highlighting}
\end{Shaded}

\includegraphics{EjemplosR_Inferencias_medias_files/figure-latex/unnamed-chunk-9-1.pdf}
Y los intervalos de Confianza de Bonferroni son:

\begin{Shaded}
\begin{Highlighting}[]
\NormalTok{b <-}\StringTok{ }\DecValTok{2}
\NormalTok{t.bonf <-}\StringTok{ }\KeywordTok{qt}\NormalTok{(}\DecValTok{1}\OperatorTok{-}\NormalTok{alpha}\OperatorTok{/}\NormalTok{(}\DecValTok{2}\OperatorTok{*}\NormalTok{b),}\DataTypeTok{df=}\NormalTok{n}\DecValTok{-1}\NormalTok{)}

\NormalTok{(l1b <-}\StringTok{ }\NormalTok{X.mean[}\DecValTok{1}\NormalTok{,}\DecValTok{1}\NormalTok{] }\OperatorTok{-}\StringTok{ }\NormalTok{t.bonf }\OperatorTok{*}\StringTok{ }\KeywordTok{sqrt}\NormalTok{(S[}\DecValTok{1}\NormalTok{,}\DecValTok{1}\NormalTok{]}\OperatorTok{/}\NormalTok{n))}
\end{Highlighting}
\end{Shaded}

\begin{verbatim}
##    closed 
## 0.5212495
\end{verbatim}

\begin{Shaded}
\begin{Highlighting}[]
\NormalTok{(u1b <-}\StringTok{ }\NormalTok{X.mean[}\DecValTok{1}\NormalTok{,}\DecValTok{1}\NormalTok{] }\OperatorTok{+}\StringTok{ }\NormalTok{t.bonf }\OperatorTok{*}\StringTok{ }\KeywordTok{sqrt}\NormalTok{(S[}\DecValTok{1}\NormalTok{,}\DecValTok{1}\NormalTok{]}\OperatorTok{/}\NormalTok{n))}
\end{Highlighting}
\end{Shaded}

\begin{verbatim}
##    closed 
## 0.6072655
\end{verbatim}

\begin{Shaded}
\begin{Highlighting}[]
\NormalTok{(l2b <-}\StringTok{ }\NormalTok{X.mean[}\DecValTok{2}\NormalTok{,}\DecValTok{1}\NormalTok{] }\OperatorTok{-}\StringTok{ }\NormalTok{t.bonf }\OperatorTok{*}\StringTok{ }\KeywordTok{sqrt}\NormalTok{(S[}\DecValTok{2}\NormalTok{,}\DecValTok{2}\NormalTok{]}\OperatorTok{/}\NormalTok{n))}
\end{Highlighting}
\end{Shaded}

\begin{verbatim}
##      open 
## 0.5596819
\end{verbatim}

\begin{Shaded}
\begin{Highlighting}[]
\NormalTok{(u2b <-}\StringTok{ }\NormalTok{X.mean[}\DecValTok{2}\NormalTok{,}\DecValTok{1}\NormalTok{] }\OperatorTok{+}\StringTok{ }\NormalTok{t.bonf }\OperatorTok{*}\StringTok{ }\KeywordTok{sqrt}\NormalTok{(S[}\DecValTok{2}\NormalTok{,}\DecValTok{2}\NormalTok{]}\OperatorTok{/}\NormalTok{n))}
\end{Highlighting}
\end{Shaded}

\begin{verbatim}
##      open 
## 0.6462806
\end{verbatim}

\begin{Shaded}
\begin{Highlighting}[]
\KeywordTok{plot}\NormalTok{(}\DecValTok{0}\NormalTok{,}\DataTypeTok{pch=}\StringTok{''}\NormalTok{,}\DataTypeTok{ylab=}\StringTok{''}\NormalTok{,}\DataTypeTok{xlab=}\StringTok{''}\NormalTok{,}\DataTypeTok{xlim=}\KeywordTok{c}\NormalTok{(}\FloatTok{0.5}\NormalTok{,}\FloatTok{0.65}\NormalTok{),}\DataTypeTok{ylim=}\KeywordTok{c}\NormalTok{(}\FloatTok{0.55}\NormalTok{,}\FloatTok{0.65}\NormalTok{))}
\KeywordTok{draw.ellipse}\NormalTok{(}\DataTypeTok{x=}\NormalTok{X.mean[}\DecValTok{1}\NormalTok{,}\DecValTok{1}\NormalTok{],}\DataTypeTok{y=}\NormalTok{X.mean[}\DecValTok{2}\NormalTok{,}\DecValTok{1}\NormalTok{],}\DataTypeTok{a=}\NormalTok{lengths[}\DecValTok{1}\NormalTok{],}\DataTypeTok{b=}\NormalTok{lengths[}\DecValTok{2}\NormalTok{],}\DataTypeTok{angle=}\NormalTok{angle,}\DataTypeTok{deg=}\OtherTok{FALSE}\NormalTok{)}

\KeywordTok{abline}\NormalTok{(}\DataTypeTok{v=}\NormalTok{l1,}\DataTypeTok{lty=}\DecValTok{3}\NormalTok{);}\KeywordTok{abline}\NormalTok{(}\DataTypeTok{v=}\NormalTok{u1,}\DataTypeTok{lty=}\DecValTok{3}\NormalTok{)}
\KeywordTok{abline}\NormalTok{(}\DataTypeTok{h=}\NormalTok{l2,}\DataTypeTok{lty=}\DecValTok{3}\NormalTok{);}\KeywordTok{abline}\NormalTok{(}\DataTypeTok{h=}\NormalTok{u2,}\DataTypeTok{lty=}\DecValTok{3}\NormalTok{)}

\KeywordTok{abline}\NormalTok{(}\DataTypeTok{v=}\NormalTok{l1b,}\DataTypeTok{lty=}\DecValTok{2}\NormalTok{);}\KeywordTok{abline}\NormalTok{(}\DataTypeTok{v=}\NormalTok{u1b,}\DataTypeTok{lty=}\DecValTok{2}\NormalTok{)}
\KeywordTok{abline}\NormalTok{(}\DataTypeTok{h=}\NormalTok{l2b,}\DataTypeTok{lty=}\DecValTok{2}\NormalTok{);}\KeywordTok{abline}\NormalTok{(}\DataTypeTok{h=}\NormalTok{u2b,}\DataTypeTok{lty=}\DecValTok{2}\NormalTok{)}
\end{Highlighting}
\end{Shaded}

\includegraphics{EjemplosR_Inferencias_medias_files/figure-latex/unnamed-chunk-10-1.pdf}

\end{document}
